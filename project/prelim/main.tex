% =========================================================================== %
% Preamble                                                                    %
% =========================================================================== %

\documentclass[dvipsnames, 12pt]{article}
\usepackage[utf8]{inputenc}

% Set paper geometry
\usepackage[letterpaper, margin=1.87cm]{geometry}

% Must include this before setting title and author
\usepackage{findlay}

\def\bpfcon{\textsc{BPFCon}}

\title{\Large \bpfcon: Towards Secure and Usable Containers with eBPF\\{\large COMP5900I Preliminary Work Document}}
\lhead{\bpfcon: Towards Secure and Usable Containers with eBPF}

\author{William Findlay}
% For acmart.cls:
%\affiliation{Carleton University}
%\email{williamfindlay@cmail.carleton.ca}

% Add bibliography:
\addbibresource{references.bib}

% To set sans serif font:
%\renewcommand{\familydefault}{\sfdefault}

% =========================================================================== %
% Document                                                                    %
% =========================================================================== %

\begin{document}

% Title page
\maketitle
\thispagestyle{empty}

\vfill
\begin{abstract}
\noindent
Containers \todo{what do I want to say about containers?}

In this paper, I present \bpfcon{}\footnote{\bpfcon{} is a working title and is
subject to change in the future.}, a novel approach to containers under the
Linux kernel, built from the ground up as a light-weight yet secure process
confinement solution for modern desktop applications. \bpfcon{} leverages eBPF,
a new Linux kernel technology, to implement fully application-transparent
confinement without the need for a privileged setuid root wrapper application.

\end{abstract}
\vfill
\vfill

\clearpage

% Reset page numbering
\pagenumbering{arabic}
\setcounter{page}{1}

% Table of Contents, List of Figures, List of Tables, List of Listings
%\begingroup
%\hypersetup{linkcolor=black}
%\tableofcontents
%\newpage
%\listoffigures
%\newpage
%\listoftables
%\newpage
%\lstlistoflistings
%\newpage
%\endgroup

% Reset page numbering
\pagenumbering{arabic}
\setcounter{page}{1}

% Uncomment for 1.5 spacing:
\onehalfspacing

\section{Introduction}

\todo{The final version of the paper will include an introduction partially
adapted from my literature review.}

\section{Background}

\todo{The background section will be mostly adapted from my literature review.}

\subsection{The Process Confinement Problem}

\todo{This will be adapted from my literature review}

\subsection{The Confinement Threat Model}

\todo{This will be adapted from my literature review}

\subsection{Low-Level Isolation Techniques}

\paragraph*{Unix Discretionary Access Control}
\todo{This will be adapted from my literature review}

\paragraph*{POSIX Capabilities}
\todo{This will be adapted from my literature review}

\paragraph*{Namespaces and Cgroups}
\todo{This will be adapted from my literature review}

\paragraph*{System Call Interposition}
\todo{This will be adapted from my literature review}

\paragraph*{Linux Security Modules}
\todo{This will be adapted from my literature review}

\subsection{Containers}

\todo{This will be adapted from my literature review}

\subsection{Extended BPF}

\section{\bpfcon{} Design Goals}
\label{sec:design}

Five specific goals informed the design of \bpfcon{}'s policy language and
enforcement mechanism, enumerated below as Design Goals \ref{d:1} to \ref{d:5}.
Note that there is a natural interplay between some of these design goals, while
others are typically perceived as being in contention (specifically usability
and security). In cases of conflict, it is essential to strike a careful balance
between each property.

\begin{enumerate}[label=\bfseries D\arabic*., ref=D\arabic*]
  \item \label{d:1} \textsc{Usability.}
    \bpfcon{}'s basic functionality should not impose unnecessary usability
    barriers on end-users.  Its policy language should be easy to understand and
    semantically meaningful to users without significant security knowledge. To
    accomplish this goal, \bpfcon{} takes some inspiration from other high-level
    policy languages for containerized applications, such as those used in
    Snapcraft \cite{snap}.

  \item \label{d:2} \textsc{Configurability.}
    It should be easy for an end-user to reconfigure policy to match their
    specific use case, without worrying about the underlying details of the
    operating system or the policy enforcement mechanism. It should be possible
    to use \bpfcon{} to restrict specific unwanted behaviour in a given
    application without needing to write a rigorous security policy from
    scratch.

  \item \label{d:3} \textsc{Transparency.}
    Containing an application using \bpfcon{} should not require modifying the
    application's source code or running the application using a privileged SUID
    (Set User ID root) binary. \bpfcon{} should be entirely agnostic to the rest
    of the system and should not interfere with its regular use.

  \item \label{d:4} \textsc{Adoptability.}
    \bpfcon{} should be adoptable across a wide variety of system configurations
    and should not negatively impact the running system. It should be possible
    to deploy \bpfcon{} in a production environment without impacting system
    stability and robustness or exposing the system to new security
    vulnerabilities. \bpfcon{} relies on the underlying properties of its eBPF
    implementation to achieve its adoptability guarantees.

  \item \label{d:5} \textsc{Security.}
    \bpfcon{} should be built from the ground up with security in mind. In
    particular, security should not be an opt-in feature and \bpfcon{} should
    adhere to the principle of least privilege \cite{saltzer1975_protection} by
    default. It should be easy to tune a \bpfcon{} policy to respond to new
    threats.
\end{enumerate}

\section{\bpfcon{} Policy}
\label{sec:policy}

\bpfcon{} policy consists of simple manifests written in YAML \cite{yaml},
a human-readable data serialization language based on key-value pairs.  Each
\bpfcon{} container is associated with a manifest, which consists of a few lines
of metadata followed by a set of \textit{rights} and \textit{restrictions}.
A \textit{right} specifies access that should be granted to a container, while
a \textit{restriction} is used revoke access. While rights and restrictions may
be combined at various levels of granularity, a restriction \textit{always}
overrides a right, without exception.

\begin{table}[htpb]
  \centering
  \caption{
    A list of accesses supported by \bpfcon{} policy, along with their
    parameters, if any, and descriptions. Square brackets denote an optional
    parameter.
    \todo{This is currently non-exhaustive, come back and revise.}
  }
  \label{tab:label}
  \begin{tabular}{llp{20em}}
  \toprule
  Access              & Parameters              & Description \\
  \midrule
  \texttt{filesystem} & Mountpoint, [Read-only] &
    Grants access at the granularity of a filesystem mountpoint.
    This access may optionally be restricted to read-only. \\
  \texttt{file}       & Pathname, Access        &
    Grants access at the granularity of individual files.
    Access may be specified as read, write, link, delete, or execute. \\
  \texttt{directory}  & Pathname, Access        &
    Grants access at the granularity of individual directories.
    Access may be specified as read, write, link, delete, or chdir. \\
  \texttt{network}    & [Interface]             &
    Grants access to network communications.
    A specific interface may optionally be specified. \\
  \texttt{tty}        & N/A                     &
    Grants access to tty devices. \\
  \texttt{graphics}   & N/A                     &
    Grants access to the graphical server. \\
  \texttt{microphone} & N/A                     &
    Grants access to the microphone. \\
  \texttt{sound}      & N/A                     &
    Grants access to the microphone. \\
  \bottomrule
  \end{tabular}
\end{table}

In accordance with the principle of least privilege
\cite{saltzer1975_protection}, \bpfcon{} implements a strict default-deny
policy, only granting access that the policy specifically declares under the
container's set of rights. The user may optionally change this behaviour and
elect to enforce a default-allow policy instead, by setting
\lstinline[language=yaml]{default: allow} in the manifest. A default-allow
policy enables the easy restriction of specific unwanted behaviour in a given
program, without worrying about the details of constructing a rigorous security
policy.

As a motivating example of \bpfcon{} security policy, consider Discord
\cite{discord}, a popular cross-platform voice chat client designed for gamers.
Discord comes with an optional feature, \enquote{Display Active Game}, which
displays whatever game the user is currently playing in their status message. To
accomplish this, the Linux Discord client periodically scans the \texttt{procfs}
filesystem to obtain a list of all running processes. While this feature may
seem innocuous at first glance, an \texttt{strace} \cite{strace} of Discord
reveals that it continually scans the process tree even when the
\enquote{Display Active Game} feature is \textit{disabled}. This behaviour
represents a gross violation of the user's privacy expectations. To rectify this
issue, a user might write a \bpfcon{} policy like the examples depicted in
\Cref{lst:discord_a} and \Cref{lst:discord_b}.

\begin{listing}[
  language=yaml,
  caption={
    A sample manifest for Discord \cite{discord} using \bpfcon{}'s more restrictive
    default-deny confinement. All accesses which are not listed under the
    container's rights are implictly denied. The explicit restriction on
    access to \texttt{procfs} overrides the access right on the root
    filesystem.
  },
  label={lst:discord_a},
  gobble=2]
  name: discord
  command: /bin/discord
  rights:
    - filesystem /
    - network
    - graphics
    - microphone
    - sound
  restrictions:
    - filesystem /proc
\end{listing}

\begin{listing}[
  language=yaml,
  caption={
    A sample manifest for Discord \cite{discord} using \bpfcon{}'s optional
    default-allow confinement.  This permits a much simpler policy that directly
    targets Discord's \texttt{procfs} scanning behaviour.
  },
  label={lst:discord_b},
  gobble=2]
  name: discord
  command: /bin/discord
  default: allow
  restrictions:
    - filesystem /proc
\end{listing}

To run a \bpfcon{} container, the user invokes \texttt{bpfcon run <name>} where
\texttt{name} is the unique container name declared in the manifest.  The
\texttt{bpfcon run} command is a thin wrapper around that target application,
whose only purpose is to invoke a special library call,
\lstinline[language=c]|bpfcon_confine()|, that marks the process group for
confinement before executing the command(s) defined in the manifest. Besides the
fact that it invokes a special library call, there is nothing special about the
\texttt{bpfcon run} wrapper---nothing is stopping a typical application from
invoking \lstinline[language=c]|bpfcon_confine()| directly, as it requires no
special privileges. In this sense, one can also think of \bpfcon{} as
a mechanism for \textit{unprivileged self-confinement}.
\Cref{sec:implementation} describes in detail the specifics of how
\bpfcon{} implements its confinement mechanism.

\section{\bpfcon{} Implementation}
\label{sec:implementation}

\section{Evaluation}

\section{Discussion}

\section{Related Work}

\section{Conclusion}

\section{Acknowledgements}

The idea for \bpfcon{} was conceived during a discussion with Anil Somayaji.
This work directly builds on my previous work, \textsc{BPFBox}
\cite{findlay20_bpfbox}, first presented in a paper co-authored with Anil
Somayaji and David Barrera.

% Uncomment for bibliography:
\clearpage
\printbibliography

\end{document}

% vim:syn=tex
