% =========================================================================== %
% Preamble                                                                    %
% =========================================================================== %

\documentclass[dvipsnames, 12pt]{article}
\usepackage[utf8]{inputenc}

% Set paper geometry
\usepackage[letterpaper, margin=1.87cm]{geometry}

% Must include this before setting title and author
\usepackage{findlay}

\title{\huge Towards Adaptive Process Confinement
Mechanisms\\{\large COMP5900I Literature Review}}

\author{William Findlay}
% For acmart.cls:
%\affiliation{Carleton University}
%\email{williamfindlay@cmail.carleton.ca}

% Add bibliography:
\addbibresource{references.bib}

% To set sans serif font:
%\renewcommand{\familydefault}{\sfdefault}

\lhead{Towards Adaptive Process Confinement Mechanisms}

% Magic to stop overfull hbox in bibliography!
\emergencystretch=1em

\ExecuteBibliographyOptions{maxbibnames=999}

\setlist[enumerate]{
  leftmargin=*
}

% =========================================================================== %
% Document                                                                    %
% =========================================================================== %

\begin{document}

% Title page
\maketitle
\thispagestyle{empty}
%\pagenumbering{roman}
%\newpage

% Table of Contents, List of Figures, List of Tables, List of Listings
%\begingroup
%\hypersetup{linkcolor=black}
%\tableofcontents
%\newpage
%\listoffigures
%\newpage
%\listoftables
%\newpage
%\lstlistoflistings
%\newpage
%\endgroup

\vfill
\begin{abstract}
\todo{Come back hither when done.}
\end{abstract}
\vfill
\vfill

\clearpage

% Reset page numbering
\pagenumbering{arabic}
\setcounter{page}{1}

% Uncomment for 1.5 spacing:
%\onehalfspacing

\section{Introduction}

Restricting unprivileged access to system resources has been a key focus of
operating systems security research since the inception of the earliest
timesharing computers in the late 1960s and early 1970s \todo{cite}. In its
earliest and simplest form, access control in operating systems meant preventing
one user from interfering with or reading the data of another user. The natural
choice for these early multi-user systems, such as Unix \cite{ritchie1973_unix},
was to build access control solutions centred around the user model---a design
choice which has persisted in modern Unix-like operating systems such as Linux,
OpenBSD, FreeBSD, and MacOS. Unfortunately, while user-centric permissions offer
at least some protection from other users, they fail entirely to protect users
from \textit{themselves} or from their own \textit{processes}.  It was long ago
recognized that finer granularity of protection is required to truly restrict
a process to its desired functionality \cite{lampson1973_a_note}. This is often
referred to as \textit{the process confinement problem} or \textit{the
sandboxing problem}.

Despite decades of work since Lampson's first proposal of the process
confinement problem in 1973 \cite{lampson1973_a_note}, it remains largely
unsolved to date \cite{crowell2013_confinement_problem}. This begs the question
as to whether our current techniques for process confinement are simply
inadequate for dealing with an evolving technical and adversarial landscape.
\todo{Rework the commented part}
%In
%this literature review, I present the status quo in process confinement, with an
%emphasis on Unix and modern Unix-like systems such as Linux.  Further, I present
%a novel taxonomy, categorizing existing process confinement mechanisms into
%\textit{maladaptive}, \textit{semi-adaptive}, and \textit{adaptive} solutions.
%Finally, I argue the case for the development and adoption of \textit{adaptive
%process confinement mechanisms}.

The rest of this paper proceeds as follows. \Cref{sec:motivation} presents the
process confinement threat model and examines the motivation for considering
process confinement from an adaptive security perspective.
\Cref{sec:traditional} discusses traditional approaches to process confinement,
both historical and state-of-the-art. \Cref{sec:automating_generation} and
\Cref{sec:automating_audit} discuss automatic and semi-automatic approaches for
policy generation and policy audit respectively. \Cref{sec:towards} discusses
moving toward truly adaptive process confinement mechanisms and argues that new
operating system technologies coupled with well-known techniques from the
anomaly detection literature may enable a paradigm shift in this direction.
\Cref{sec:conclusion} concludes.

\section{Motivation}
\label{sec:motivation}

\subsection{The Process Confinement Threat Model}

To understand why process confinement is a desirable goal in operating system
security, we must first identify the credible threats to system stability and
security that process confinement addresses. To that end, I first describe three
attack vectors (\crefrange{a:1}{a:3}), followed by three attack goals
(\crefrange{g:1}{g:3}) which highlight just a few of the credible threats posed
by unconfined processes running on a given host.

\begin{enumerate}[label=\bfseries A\arabic*., ref=A\arabic*, labelindent=2em]
    \item \label{a:1} \textsc{Compromised processes.} Unconfined running
    processes have classically presented a valuable target for attacker
    exploitation. With the advent of the Internet, web-facing processes which
    handle untrusted user input are especially vulnerable, particularly as they
    often run with heightened privileges \cite{cohen1996_secure}. The attacker
    may send specially crafted input to the target application, hoping to
    compromise its control flow integrity via a classic buffer overflow,
    return-oriented programming \cite{shacham2007_rop}, or some other means. The
    venerable Morris Worm, regarded as the first computer worm on the Internet,
    exploited a classic buffer overflow vulnerability in the \texttt{fingerd}
    service for Unix, as well as a development backdoor left in the
    \texttt{sendmail} daemon \cite{spafford1989_morris}. In both cases, proper
    process confinement would have eliminated the threat entirely by preventing
    the compromised programs from impacting the rest of the system.

    \item \label{a:2} \textsc{Semi-honest software.} Here, I define semi-honest
    software as that which appears to perform its desired functionality, but
    which additionally may perform some set of unwanted actions without the
    user's knowledge. Without putting a proper, external confinement mechanism
    in place to restrict the behaviour of such an application, it may continue
    to perform the undesired actions ad infinitum, so long as it remains
    installed on the host. As a topical example, an \texttt{strace} of the
    popular Discord \cite{discord} voice communication client on Linux reveals that
    it repeatedly scans the process tree and reports a list of \textit{all applications}
    running on the system, even when the \enquote{display active game} feature
    is turned \textbf{off}. This represents a clear violation of the user's
    privacy expectations.

    \item \label{a:3} \textsc{Malicious software.} In contrast to semi-honest
    software, malicious software is that which is expressly designed and
    distributed with malicious intent. Typically this software would be
    downloaded by an unsuspecting user either through social engineering
    (e.g.~fake antivirus scams) or without the user's knowledge (e.g.~a drive-by
    download attack). It would be useful to provide the user with a means of
    running such potentially untrustworthy applications in a sandbox so that
    they cannot damage the rest of the system.
\end{enumerate}

\begin{enumerate}[label=\bfseries G\arabic*., ref=G\arabic*, labelindent=2em]
    \item \label{g:1} \textsc{Installation of backdoors/rootkits.} Potentially
    the most dangerous attack goal in the exploitation of unconfined processes
    is the establishment of a backdoor on the target system.  A backdoor needn't
    be sophisticated---for example, installing the attacker's RSA public key in
    \texttt{ssh}'s list of authorized keys would be sufficient---however the
    most sophisticated backdoors may result in permanent and virtually
    undetectable escalation of privilege. For instance, a sophisticated attacker
    with sufficient privileges may load a \textit{rootkit}
    \cite{beegle2007_rootkit} into the operating system kernel, at which point
    she has free reign over the system in perpetuity (unless the rootkit is
    somehow removed or the operating system is reinstalled).

    \item \label{g:2} \textsc{Information leakage.} An obvious goal for attacks
    on unconfined processes (and indeed the focus of the earliest literature on
    process confinement \cite{lampson1973_a_note}) is information leakage. An
    adversary may attempt to gain access personal information or other sensitive
    data such as private keys, password hashes, or bank credentials. Depending
    on the type of information, an unauthorized party may not even necessarily
    require elevated privileges to access it---for instance, no special
    privileges are required to leak the list of processes running on a Linux
    system, as in the case of Discord \cite{discord} highlighted above.

    \item \label{g:3} \textsc{Tampering and denial of service.} \todo{Write this}
\end{enumerate}

\todo{Talk about how the internet has exacerbated this problem}

\subsection{Defining Adaptive Process Confinement}

Here, I define adaptive process confinement mechanisms as those which greatly
help defenders confine their processes and are robust in the presence of
attacker innovation. Roughly, this definition can be broken down into the
following properties:
\begin{enumerate}[label=\bfseries P\arabic*., ref=P\arabic*, labelindent=2em]
    \item \label{p:1} \textsc{Robustness to attacker innovation.} An adaptive
    process confinement mechanism should continue to protect the host system,
    even in the presence of attacker innovation. That is, it should be resistant
    to an adaptive adversary.

    \item \label{p:2} \textsc{Low adoption effort.} An adaptive process
    confinement mechanism should require minimal effort to adopt on a variety of
    system configurations.

    \item \label{p:3} \textsc{High reconfigurability.} An adaptive process
    confinement mechanism should be highly reconfigurable based on the needs of
    the end user and the environment in which it is running. This
    reconfiguration could either be automated, semi-automated, or manual, but
    should not impose significant adoptability barriers (c.f.~\ref{p:2}).

    \item \label{p:4} \textsc{Transparency.} An adaptive process confinement
    mechanism should be as transparent to the end user as possible. It should
    not get in the way of ordinary system functionality.

    \item \label{p:5} \textsc{Usability.} An adaptive process confinement
    mechanism should maximize its usability such that it is usable by largest
    and most diverse set of defenders possible. In particular, it should not
    require significant computer security expertise from its users.
\end{enumerate}
Ideally, an adaptive process confinement mechanism should have most---if not
all---of the above properties.

\section{Traditional Process Confinement Approaches}
\label{sec:traditional}

\begin{figure}[htpb]
    \centering
    \includegraphics[width=0.6\linewidth]{figs/high-level.pdf}
    \caption{
        The basic architecture of containerized package management solutions for
        Linux, such as Snapcraft \cite{snap}, Flatpak \cite{flatpak}, and Docker
        \cite{docker}. Package maintainers write high-level, coarse-grained
        package manifests, which are then compiled into policy for lower-level
        process confinement mechanisms to enforce.
    }%
    \label{fig:containerized}
\end{figure}

\section{Automating Policy Generation}
\label{sec:automating_generation}

\section{Automating Policy Audit}
\label{sec:automating_audit}

\section{Towards Truly Adaptive Process Confinement}
\label{sec:towards}

\subsection{Anomaly Detection Techniques}

\subsection{Extended BPF}

\section{Conclusion}
\label{sec:conclusion}


% Uncomment for bibliography:
\nocite{*} % TODO: Comment this out when done
\clearpage
\printbibliography

\end{document}

% vim:syn=tex
