% =========================================================================== %
% Preamble                                                                    %
% =========================================================================== %

\documentclass[dvipsnames, 12pt]{article}
\usepackage[utf8]{inputenc}

% Set paper geometry
\usepackage[letterpaper, margin=1.87cm]{geometry}

% Must include this before setting title and author
\usepackage{findlay}

\title{\huge Towards Adaptive Process Confinement
Mechanisms\\{\large COMP5900I Literature Review}}

\author{William Findlay}
% For acmart.cls:
%\affiliation{Carleton University}
%\email{williamfindlay@cmail.carleton.ca}

% Add bibliography:
\addbibresource{references.bib}

% To set sans serif font:
%\renewcommand{\familydefault}{\sfdefault}

\lhead{Towards Adaptive Process Confinement Mechanisms}

% Magic to stop overfull hbox in bibliography!
\emergencystretch=1em

\ExecuteBibliographyOptions{maxbibnames=999}

\setlist[enumerate]{
  leftmargin=*
}

% =========================================================================== %
% Document                                                                    %
% =========================================================================== %

\begin{document}

% Title page
\maketitle
\thispagestyle{plain}
%\pagenumbering{roman}
%\newpage

% Table of Contents, List of Figures, List of Tables, List of Listings
%\begingroup
%\hypersetup{linkcolor=black}
%\tableofcontents
%\newpage
%\listoffigures
%\newpage
%\listoftables
%\newpage
%\lstlistoflistings
%\newpage
%\endgroup

% Reset page numbering
\pagenumbering{arabic}
\setcounter{page}{1}

\clearpage

\begin{abstract}
\todo{Come back hither when done.}
\end{abstract}

% Uncomment for 1.5 spacing:
%\onehalfspacing

\section{Introduction}

Restricting unprivileged access to system resources has been a key focus of
operating systems security research since the inception of the earliest
timesharing computers in the late 1960s and early 1970s \todo{cite}. In its
earliest and simplest form, access control in operating systems meant preventing
one user from interfering with or reading the data of another user. The natural
choice for these early multi-user systems, such as Unix \cite{ritchie1973_unix},
was to build access control solutions centred around the user model---a design
choice which has persisted in modern Unix-like operating systems such as Linux,
OpenBSD, FreeBSD, and MacOS. Unfortunately, while user-centric permissions offer
at least some protection from other users, they fail entirely to protect users
from \textit{themselves} or from their own \textit{processes}.  It was long ago
recognized that finer granularity of protection is required to truly restrict
a process to its desired functionality \cite{lampson1973_a_note}. This is often
referred to as \textit{the process confinement problem} or \textit{the
sandboxing problem}.

\subsection{The Process Confinement Threat Model}

To understand why process confinement is a desirable goal in operating system
security, we must first identify the credible threats to system stability and
security that process confinement addresses. To that end, I first describe three
attack vectors (\crefrange{a:1}{a:3}), followed by five attack goals
(\crefrange{g:1}{g:5}) which highlight just a few of the credible threats posed
by unconfined processes running on a given host.

\begin{enumerate}[label=\bfseries A\arabic*., ref=A\arabic*, labelindent=2em]
    \item \label{a:1} \textsc{Compromised processes.} \todo{Write this}

    \item \label{a:2} \textsc{Semi-honest software.} \todo{Write this}

    \item \label{a:3} \textsc{Malicious software.} \todo{Write this}
\end{enumerate}

\begin{enumerate}[label=\bfseries G\arabic*., ref=G\arabic*, labelindent=2em]
    \item \label{g:1} \textsc{Installation of backdoors/rootkits.} \todo{Write this}

    \item \label{g:2} \textsc{Compromise of trusted computing base.} \todo{Write this}

    \item \label{g:3} \textsc{Unauthorized access to files.} \todo{Write this}

    \item \label{g:4} \textsc{Denial of service.} \todo{Write this}

    \item \label{g:5} \textsc{Theft of computational resources.} \todo{Write this}
\end{enumerate}

\todo{Talk about how the internet has exacerbated this problem}

\subsection{The Case for Adaptive Process Confinement}

Despite decades of work since Lampson's first proposal of the process
confinement problem in 1973 \cite{lampson1973_a_note}, the process confinement
problem remains largely unsolved \cite{crowell2013_confinement_problem}. This
begs the question as to whether our current techniques for process confinement
are simply inadequate for dealing with an evolving technical and adversarial
landscape. In this literature review, I present the status quo in process
confinement, with an emphasis on Unix and modern Unix-like systems such as
Linux.  Further, I present a novel taxonomy, categorizing existing process
confinement mechanisms into \textit{maladaptive}, \textit{semi-adaptive}, and
\textit{adaptive} solutions. Finally, I argue the case for the development and
adoption of \textit{adaptive process confinement mechanisms}.

Here, I define adaptive process confinement mechanisms as those which greatly
help defenders confine their processes and are robust in the presence of
attacker innovation. Roughly, this definition can be broken down into the
following properties:
\begin{enumerate}[label=\bfseries P\arabic*., ref=P\arabic*, labelindent=2em]
    \item \label{p:1} \scshape{Robustness to attacker innovation.} \todo{Write this}
    \item \label{p:2} \scshape{Low adoption effort.} \todo{Write this}
    \item \label{p:3} \scshape{High reconfigurability.} \todo{Write this}
    \item \label{p:4} \scshape{Transparency.} \todo{Write this}
    \item \label{p:5} \scshape{Usability.} \todo{Write this}
\end{enumerate}
Ideally, an adaptive process confinement mechanism should have most---if not
all---of the above properties.

\subsection{Outline}

The rest of this paper proceeds as follows. \todo{List sections and what is in them.}

\section{Traditional Process Confinement Approaches}

\section{Automating Policy Generation}

\section{Automating Policy Audit}

\section{Towards Truly Adaptive Process Confinement}

\subsection{Anomaly Detection Techniques}

\subsection{Extended BPF}

\section{Conclusion}


% Uncomment for bibliography:
\nocite{*} % TODO: Comment this out when done
\clearpage
\printbibliography

\end{document}

% vim:syn=tex
