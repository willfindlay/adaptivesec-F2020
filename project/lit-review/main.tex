% =========================================================================== %
% Preamble                                                                    %
% =========================================================================== %

\documentclass[dvipsnames, 12pt]{article}
\usepackage[utf8]{inputenc}

% Set paper geometry
\usepackage[letterpaper, margin=1.87cm]{geometry}

% Must include this before setting title and author
\usepackage{findlay}

\title{\huge Towards Adaptive Process Confinement
Mechanisms\\{\large COMP5900I Literature Review}}

\author{William Findlay}
% For acmart.cls:
%\affiliation{Carleton University}
%\email{williamfindlay@cmail.carleton.ca}

% Add bibliography:
\addbibresource{references.bib}

% To set sans serif font:
%\renewcommand{\familydefault}{\sfdefault}

\lhead{Towards Adaptive Process Confinement Mechanisms}

% Magic to stop overfull hbox in bibliography!
\emergencystretch=1em

\ExecuteBibliographyOptions{maxbibnames=999}

% =========================================================================== %
% Document                                                                    %
% =========================================================================== %

\begin{document}

% Title page
\maketitle
\thispagestyle{plain}
%\pagenumbering{roman}
%\newpage

% Table of Contents, List of Figures, List of Tables, List of Listings
%\begingroup
%\hypersetup{linkcolor=black}
%\tableofcontents
%\newpage
%\listoffigures
%\newpage
%\listoftables
%\newpage
%\lstlistoflistings
%\newpage
%\endgroup

% Reset page numbering
\pagenumbering{arabic}
\setcounter{page}{1}

\begin{abstract}
\todo{Come back hither when done.}
\end{abstract}

% Uncomment for 1.5 spacing:
\onehalfspacing

\section{Introduction}

Restricting unprivileged access to system resources has been a key focus of
operating systems security research since the inception of the earliest
timesharing computers in the late 1960s and early 1970s
\cite{lampson_1973_a_note, graham_1968_protection}. With the advent of the
Internet and multi-tenant cloud computing, the problem of protecting hosts from
their own applications has been further exacerbated.

Despite decades of work, the process confinement problem remains largely
unsolved. Traditionally access to security-sensitive resources was controlled by
the reference monitor mechanisms built into the operating system.

\subsection{The Process Confinement Threat Model}

\todo{Threat Vectors}
\begin{enumerate}[label=\bfseries T\arabic*.]
    \item \textsc{Malicious software.}

    \item \textsc{Semi-honest software.}

    \item \textsc{Compromised processes.}
\end{enumerate}

\todo{Attack Goals}
\begin{enumerate}[label=\bfseries A\arabic*.]
    \item \textsc{Installation of backdoors/rootkits.}

    \item \textsc{Compromise of trusted computing base.}

    \item \textsc{Unauthorized access to files.}

    \item \textsc{Denial of service.}

    \item \textsc{Theft of computational resources.}
\end{enumerate}

\subsection{Outline}

The rest of this paper proceeds as follows. \todo{List sections and what is in them.}

\section{Traditional Process Confinement Approaches}

\section{Automating Policy Generation}

\section{Automating Policy Audit}

\section{Integrating System State into Process Confinement}

\subsection{Anomaly Detection}

\subsection{Extended BPF}

\section{Conclusion}


% Uncomment for bibliography:
\nocite{*} % TODO: Comment this out when done
\clearpage
\printbibliography

\end{document}

% vim:syn=tex
