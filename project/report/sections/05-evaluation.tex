\section{Evaluation}
\label{sec:eval}

\subsection{Security}

According the Anderson's reference monitor model \cite{anderson1973_planning_study}, a secure reference validation mechanism must satisfy the properties of \textit{complete mediation}, \textit{tamper resistance}, and \textit{verifiability}. Complete mediation refers to the property of mediating all possible accesses to a security-sensitive resource with no ability to bypass the reference monitor. Tamper resistance means that it should not be possible for an external actor to interfere with or change the reference monitor's behaviour. Finally, verifiability refers to the ability to formally or informally verify the correctness of a reference monitor. Generally speaking, this means that the underlying implementation should be terse and simple enough that its efficacy can be reasoned about in practice \cite{jaeger2011_reference_monitor}. As a least-privilege implementation, \bpfcontain{} must adhere to all three of these properties (at least within the scope of container security) to be considered secure. I discuss each in the paragraphs that follow.

\paragraph*{Complete Mediation}

Recall that most of \bpfcontain{}'s policy enforcement mechanism leverages the Linux Security Modules framework exposed by the kernel. We assume that the property of complete mediation holds for the LSM framework itself. Therefore, we can say that \bpfcontain{} achieves complete mediation insofar as its LSM-level policy is concerned. Other aspects of \bpfcontain{}'s policy enforcement, such as its instrumentation of the \texttt{commit\_creds} function in the kernel, serve only to complement its LSM-level policy rather than replace it. Such extensions provide additional kernel-level hardening against attacks mounted from containers and, thus, increase the strength of \bpfcontain{}'s complete mediation guarantees. In summary, we can say that \bpfcontain{}'s complete mediation at least reduces to that of the LSM framework itself, and extends it in the best case.

\paragraph*{Tamper Resistance}

Due to \bpfcontain{}'s eBPF-based implementation, the presence of the \texttt{bpf(2)} system call on the host system introduces certain risks that must be carefully managed to achieve tamper resistance. The primary risk involves unwanted modification of \bpfcontain{}'s policy maps. To mitigate this risk, \bpfcontain{} includes specific logic in its LSM probe on the \texttt{bpf(2)} system call itself, preventing any userspace process besides the \bpfcontain{} daemon itself from directly accessing or modifying its policy maps. Additionally, maps responsible for managing process and container state are restricted using the \texttt{BPF\_F\_READONLY} flag, meaning that they can only be modified from within \bpfcontain{}'s LSM probes, and never via the \texttt{bpf(2)} system call \cite{linux_bpf}. Even without these safeguards, the kernel gates access to eBPF maps with the CAP\_SYS\_ADMIN capability. This requirement means that a process would effectively require root privileges before accessing any map on the system \cite{linux_bpf}.

With the integrity of \bpfcontain{}'s policy and state maps covered, all that remains is to show that its maps and eBPF programs are also protected from being removed or replaced in the kernel. For this, we rely on the inherent properties of how the kernel manages the lifecycle of eBPF maps and programs. For each \textit{BPF object}\footnote{Think of this as an umbrella term covering both eBPF maps and programs.}, the kernel maintains a reference counter, keeping track of the number of open file descriptors referring to the object. The kernel only detaches and cleans up a given BPF object once its reference counter reaches zero \cite{starovoitov2018_lifetime}. Thus, so long as the \bpfcontain{} daemon remains running and does not close these open file descriptors, its programs and maps are guaranteed to be loaded in the kernel. As an extra precaution, \bpfcontain{} also pins all of its maps and programs to a special filesystem called \texttt{bpffs}. Pinning an object to \texttt{bpffs} increments its reference counter by one until an \texttt{unlink(2)} system call removes the pinned inode \cite{starovoitov2018_lifetime}. Thus, \bpfcontain{}'s eBPF objects persist even when \bpfcontain{} is killed, for example, by a malicious privileged process. To protect its pinned objects, \bpfcontain{} instruments an LSM probe preventing any other process from calling \texttt{unlink(2)} on its pinned file descriptors.

\paragraph*{Verifiability}

\bpfcontain{} has a comparatively small codebase (well under 2000 lines of kernelspace code) compared to conventional LSM implementations such as SELinux or AppArmor. Much of its functionality, such as policy management, is offloaded entirely to userspace, further reducing the amount of kernelspace code that must be verified. Finally, because of its eBPF implementation, \bpfcontain{}'s kernelspace code must pass automated, formal verification checks before it can even be loaded into the kernel. Because of these checks, we can be confident that \bpfcontain{}'s enforcement mechanism does not suffer from the memory safety issues and other security-critical bugs that plague conventional C programs. Therefore, we can say that \bpfcontain{} achieves the verifiability property.

\subsubsection{Evaluating \bpfcontain{}'s Security in Practice}

Evaluation of \bpfcontain{}'s security in practice will require demonstration that a correctly configured \bpfcontain{} policy can defend against container exploitation. This will involve three phases:
\begin{itemize}
  \item Identifying relevant exploits for a set of container images;
  \item Constructing a security policy for these images; and
  \item Testing that the security policy effectively prevents the exploits without affecting the desired functionality of the container image.
\end{itemize}
Xin \etal~\cite{xin2018_container_security} presented a rigorous dataset of container exploits up to the year 2018, which can bootstrap the testing dataset for \bpfcontain{} and serve as a strong basis for comparison against existing container security measures. This dataset will then be augmented with additional CVEs from 2019 and 2020.

In addition to container-specific exploits, we can also test \bpfcontain{}'s efficacy for ad-hoc confinement by constructing relevant \bpfcontain{} policy for commodity applications and testing against known CVEs or a set of undesired behaviour for these applications. For instance, these tests might involve verifying the prevention of a code execution exploit in a web server or stopping Discord's unwanted procfs scanning behaviour using the policy depicted in \Cref{sub:case_studies}.

\subsection{Usability}

Informally, the usability argument for \bpfcontain{} is obvious. By exposing a high-level policy language with semantics that will be familiar to end-users, \bpfcontain{} makes it easy to write secure policies that pertain to a container's specific needs. Higher-level rules may be combined with lower-level rules for finer-grained enforcement where required. For use cases that focus on limiting specific behaviours in a confined application, the default-allow policy option provides a convenient way to write ad-hoc policy without worrying about the underlying details. For instance, the default allow policy for restricting Discord's procfs scanning behaviour (c.f.~\Cref{lst:discord_b} in \Cref{sub:case_studies}) is only nine lines long, and consists of three actual policy rules.

To properly evaluate \bpfcontain{}'s usability, we must establish a strong basis for comparing it to other confinement mechanisms, container-focused and otherwise. To that end, a user study will be conducted, wherein end-users are requested to evaluate \bpfcontain{} based on its perceived security level, ease of policy authorship, and alignment with user expectations. Before providing an evaluation, a participant would use \bpfcontain{} for some time to confine commonly-used applications such as text editors, voice chat clients, and web browsers. Because \bpfcontain{} targets both ad-hoc confinement and package management as potential use cases, it may also be valuable to conduct a similar study asking application authors to write \bpfcontain{} policy for their own software projects.  Participants would be asked to provide the same evaluation for an existing policy mechanism, such as AppArmor, to establish a basis for comparison with other policy mechanisms.
%In particular, we might consider participant agreement with the following statements, evaluated on a Likert scale:
%\begin{enumerate}[label=\bf Q\arabic*.]
%  \item I was able to write the policy I needed for my use case.
%  \item I was confident that my confined applications were more secure.
%  \item \todo{Figure out more questions}
%\end{enumerate}

\subsection{Performance}

To establish \bpfcontain{}'s performance overhead on the running system, we can employ various benchmarking tests that target the operations \bpfcontain{} interposes on. The same benchmarks will be run on a standard Docker configuration without any confinement and a Docker configuration using Docker's basic AppArmor and seccomp policy to establish a baseline for comparison. Since Linux 5.1, the kernel now also supports runtime benchmarking of eBPF programs, providing a means of directly measuring the overhead associated with a given probe \cite{starovoitov2019_bpf_benchmark}. Using this data, we can isolate each eBPF program's precise overhead and corroborate the other tests' results. \Cref{tab:perf_tests} describes the proposed benchmarks in detail. For reproducibility, tests will be conducted for multiple trials (separately including and excluding the kernel's built-in eBPF benchmarks) on a dedicated host system using a KVM virtual machine.

{
\small
\begin{longtable}[c]{lp{25em}}
  \caption{
    Benchmarking tests that will be used to evaluate the performance overhead of \bpfcontain{}.
  }
  \label{tab:perf_tests}\\
  \toprule
  Test               & Description \\
  \midrule
  \endfirsthead
  Phoronix OSBench   & The Phoronix OSBench benchmarking suite \cite{phoronix} is a macro-benchmarking suite that measures basic operating system functionality, such as program execution, filesystem access, and memory allocations. \\

  System Calls       & This test will involve writing micro-benchmarking programs whose only purpose is to continually invoke system calls interposed on by \bpfcontain{}'s LSM probes. Some particularly relevant system calls might be \texttt{read}, \texttt{write}, \texttt{mmap}, \texttt{clone}, and \texttt{execve}. The programs will measure the average time taken to perform each system call. \\

  Kernel eBPF Benchmarks    & Linux 5.1 added a tunable sysctl parameter for enabling system-wide benchmarking of eBPF programs \cite{starovoitov2019_bpf_benchmark}. Using this parameter, we can directly measure the overhead imposed by \bpfcontain{}'s eBPF programs during regular system operation. Additionally, these built-in benchmarks can be used to corroborate findings in the formal tests outlined above. \\
  \bottomrule
\end{longtable}
}
