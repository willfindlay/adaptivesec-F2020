\section{Related Work}

Several researchers \cite{sultan2019_container_security,xin2018_container_security,mullinix2020_security_measures} have examined various aspects of the container security, across a variety of container management platforms and confinement mechanisms. Sultan \etal~\cite{sultan2019_container_security} examined the container security landscape in detail, identifying strengths, weaknesses, patterns in academic literature,  and promising future work opportunities. Their recommendations included work towards a container-specific LSM \cite{sultan2019_container_security}, which inspired the research direction for \bpfcontain{}. Xin \etal~\cite{xin2018_container_security} presented a detailed taxonomy of container security exploits and analyzed the security of container management platforms against their exploit database. Mullinix \etal~\cite{mullinix2020_security_measures} presented a detailed analysis of Docker security and its underlying mechanisms along with the state of the art solutions in academia for measuring and hardening Docker security.

Vulnerability analysis of container images \cite{shu2017_image_vuln,kwon2020_divds,brady2020_docker_cloud} has proved a lucrative technique for identifying areas of weakness in container configurations. Shu \etal~\cite{shu2017_image_vuln} presented their DIVA framework for automatic Docker Hub image vulnerability analysis and aggregated vulnerability data from over 350,000 Docker Hub images. In particular, they found that Docker images contained an average of 180 security vulnerabilities and that these vulnerabilities often propagate between parent and child images \cite{shu2017_image_vuln}. Kwon and Lee \cite{kwon2020_divds} used a similar technique in DIVDS, extracting vulnerabilities from container images and offering an interface to compare vulnerability severity and optionally add specific vulnerabilities to an allowlist. Brady \etal~\cite{brady2020_docker_cloud} applied image vulnerability scanning to a continuous integration pipeline to identify container vulnerabilities in production deployments.

Other approaches consider ways to harden container management platforms directly, using existing Linux security features. Chen \etal~\cite{chen2019_container_dos} proposed a framework for mitigating denial of service attacks against the host system mounted from containers, using a combination of cgroups and kernel-module-based enforcement to limit resource consumption. While such cgroup-based methods effectively restrict resource-based denial of service attacks, they are insufficient for implementing least-privilege. To simplify system call filtering rules and reduce overprivileged access to the host system, Lei \etal~\cite{lei2017_speaker} introduced SPEAKER to partition a container's seccomp-bpf profile into multiple execution phases. The critical insight that informed their approach was that a container's setup phase and main work loop often involve disparate sets of system calls \cite{lei2017_speaker}. Xin \etal~\cite{xin2018_container_security} proposed patching critical functions in the kernel such as \texttt{commit\_creds} to be container-aware to mitigate the threat of kernel privilege escalation exploits mounted from containers.

Many least-privilege enforcement mechanisms for container security rely on Linux Security Modules for mandatory access control. Loukidis \etal~\cite{loukidis2018_dockersec} proposed a mechanism for automatically deriving per-container AppArmor policy based on image characteristics and runtime information gathered from individual containers. Based on the observation that different containers often have disparate security requirements, Sun \etal~\cite{sun2018_security_namespace} proposed adopting a new security namespace to allow specific containers to load their own LSM implementations, independent of the rest of the system. Citing their work as a good starting point, Sultan \etal~\cite{sultan2019_container_security} proposed that further research should be dedicated to the notion of a container-specific LSM. \bpfcontain{} represents another step towards such a container-specific LSM implementation.

\bpfcontain{} is not the first research project to propose exposing LSM hooks to userspace through eBPF. Landlock \cite{landlockio,landlock_patch} is an experimental Linux Security Module presented by Salaun to expose a subset of LSM hooks to unprivileged userspace programs. Under Landlock, userspace programs write and load eBPF programs into the kernel to filter their accesses. Unfortunately, the community has since recognized that allowing unprivileged processes to load eBPF programs into the kernel is fundamentally insecure, regardless of any limitations imposed on program type and functionality \cite{corbet2019_krsi}. Thus, Landlock has not been merged into the mainline kernel and will likely remain out-of-tree going forward. Unlike Landlock, Singh's KRSI \cite{corbet2019_krsi,singh2019_krsi} allows privileged users to attach eBPF programs to LSM hooks. Since KRSI does not require unprivileged processes to load and manage eBPF programs, it does not suffer the same fundamental security issues that have detracted from Landlock.

While KRSI serves as the infrastructure for implementing LSM programs in eBPF, developers must still provide their own implementations for any eBPF LSM hooks they wish to use. Findlay \etal~\cite{findlay2020_bpfbox} introduced bpfbox as the first full process confinement mechanism using these eBPF LSM hooks. Unlike \bpfcontain{}, bpfbox was focused on a generic sandboxing use case rather than a container-specific LSM implementation. \bpfcontain{} can be thought of as a direct successor to bpfbox, taking lessons learned from its development as a generic sandboxing framework and applying them specifically to the realm of containers.
