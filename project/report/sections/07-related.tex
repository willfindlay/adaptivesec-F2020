\section{Related Work}

Several researchers \cite{sultan2019_container_security,xin2018_container_security,mullinix2020_security_measures} have examined various aspects of the container security, across a variety of container management platforms and confinement mechanisms. Sultan \etal~\cite{sultan2019_container_security} examined the container security landscape in detail, identifying strengths, weaknesses, patterns in academic literature,  and promising future work opportunities. Their recommendations included work towards a container-specific LSM \cite{sultan2019_container_security}, which inspired the research direction for \bpfcontain{}. Xin \etal~\cite{xin2018_container_security} presented a detailed taxonomy of container security exploits and analyzed the security of container management platforms against their exploit database. Mullinix \etal~\cite{mullinix2020_security_measures} presented a detailed analysis of Docker security and its underlying mechanisms along with the state of the art solutions in academia for measuring and hardening Docker security.

Vulnerability analysis of container images \cite{shu2017_image_vuln,kwon2020_divds,brady2020_docker_cloud} has proved a lucrative technique for identifying areas of weakness in container configurations. Shu \etal~\cite{shu2017_image_vuln} presented their DIVA framework for automatic Docker Hub image vulnerability analysis and aggregated vulnerability data from over 350,000 Docker Hub images. In particular, they found that Docker images contained an average of 180 security vulnerabilities and that these vulnerabilities often propagate between parent and child images \cite{shu2017_image_vuln}. Kwon and Lee \cite{kwon2020_divds} used a similar technique in DIVDS, extracting vulnerabilities from container images and offering an interface to compare vulnerability severity and optionally add specific vulnerabilities to an allowlist. Brady \etal~\cite{brady2020_docker_cloud} applied image vulnerability scanning to a continuous integration pipeline to identify container vulnerabilities in production deployments.

Other approaches consider ways to harden container management platforms directly, using existing Linux security features. Chen \etal~\cite{chen2019_container_dos} proposed a framework for mitigating denial of service attacks against the host system mounted from containers, using a combination of cgroups and kernel-module-based enforcement to limit resource consumption. While such cgroup-based methods effectively restrict resource-based denial of service attacks, they are insufficient for implementing least-privilege. To rectify overprivileged access to system calls, Lei \etal~\cite{lei2017_speaker} introduced SPEAKER to partition a container's seccomp-bpf profile into multiple execution phases.

% Other approaches to improve container security
\todo{Lei \etal~\cite{lei2017_speaker} proposed SPEAKER which partitions a container's seccomp-bpf profile into two phases.}
\todo{Loukidis \etal~\cite{loukidis2018_dockersec} introduced Docker-Sec as a fully automated mechanism for enhancing the security of Docker containers}
\todo{mention Xin \etal~\cite{xin2018_container_security} proposed hardening Linux's \texttt{commit\_creds} kernel function against privilege escalation exploits for processes confined by a namespace}

% LSM stuff
\todo{Sun \etal~\cite{sun2018_security_namespace} proposed security namespaces, allowing individual containers to opt-in to a specific LSM without impacting the rest of the system}
\todo{Findlay \etal~\cite{findlay2020_bpfbox} presented bpfbox, which uses eBPF LSM probes to implement a process confinement mechanism for Linux applications. \bpfcontain{} can be thought of as a natural extension to bpfbox, taking the generic approach to runtime security and focusing it specifically on container-specific applications.}
\todo{Landlock LSM~\cite{landlockio,landlock_patch} is an attempt at providing unprivileged LSMs to userspace as BPF programs, but it has since been recognized that is not possible to securely allow userspace to load arbitrary BPF programs into the kernel, and landlock remains an out-of-tree patch}

