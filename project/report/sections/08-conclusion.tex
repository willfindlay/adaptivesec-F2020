\section{Conclusion}

This paper has presented \bpfcontain{}, a novel least-privilege implementation for container security that leverages the power of eBPF to provide safe and flexible policy enforcement at the container level. Due to its implementation as an eBPF-based enforcement mechanism, \bpfcontain{} both hardens the kernel against privilege escalation exploits and secures the container-host boundary against undesired (and potentially dangerous) misuse of system resources. Further, \bpfcontain{} exposes a simple YAML-based policy configuration language to userspace that conforms with the semantics of existing container management mechanisms. This policy language is designed to be highly configurable and to support ad-hoc confinement use cases through high-level policy rules and optional default-allow enforcement.

In the future, formal evaluation is necessary to establish \bpfcontain{}'s usability, security, and performance overhead. Direct comparison against other confinement mechanisms is essential to determine \bpfcontain{}'s relative efficacy and encourage its adoption as a new standard for secure containers. Other promising avenues for future work include the addition of full virtualization support, direct integration with OCI-compliant container management systems, and further investigation into how else we can harden the kernel using kprobe-based instrumentation of critical code paths.
