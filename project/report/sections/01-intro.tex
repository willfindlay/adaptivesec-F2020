\section{Introduction}

Containers offer a light-weight alternative to traditional hypervisor-based virtualization techniques by leveraging a thin virtualization layer provided by the operating system and sharing the operating system kernel and resources \cite{sultan2019_container_security,sun2018_security_namespace}. Thanks to their low overhead, high scalability, and composability, containers are seeing increased industry adoption, particularly in multi-tenant platforms such as the cloud \todo{cite}. Despite these performance and convenience advantages, containers suffer from weaker security than traditional virtualization techniques, in part because they must share the underlying operating system kernel \cite{sultan2019_container_security,sun2018_security_namespace,xin2018_container_security}. Another major factor impacting the security of containers is the relatively lax attitude that many container management frameworks take toward the goal of least-privilege \cite{sultan2019_container_security}. Popular container frameworks such as Docker \cite{docker} provision overly-permissive default access, rely on a complex and often ill-suited suite of security mechanisms provided by the host system, and support insecure configuration options \cite{docker,sultan2019_container_security,xin2018_container_security,findlay2020_bpfbox}.

\todo{Talk about \bpfcontain{} and what it does}

\todo{List contributions: container-specific LSM implementation, security-first approach to containers, least-privilege at the LSM level and hardening against kernel exploitation using kprobes}

In summary, I offer the following contributions:
\begin{itemize}
  \item I present \bpfcontain{} a novel least-privilege enforcement mechanism for containers, leveraging eBPF programs attached to LSM hooks in the kernel. \bpfcontain{} improves container security by offering a container-specific LSM that can be dynamically loaded at runtime. Using eBPF, it also directly integrates with other parts of the operating system kernel, hardening the host against kernel privilege escalation attacks mounted from a container.

  \item I propose a simple YAML-based policy language that offers optional layers of granularity to meet various use cases and the needs of multiple stakeholders. The policy language is expressive enough that it can be used to confine individual system resources, yet simple enough that it affords ad-hoc confinement use cases.

  \item Using the \bpfcontain{} prototype as a proof of concept, I argue that it can be integrated directly with existing container management frameworks without modifying their source code, offering significant security advantages over traditional approaches to container least-privilege. I present opportunities for future work that will allow \bpfcontain{} to both significantly streamline its policy language while simultaneously improving upon the status quo in confinement.
\end{itemize}

\todo{The rest of this paper proceeds as follows...}
