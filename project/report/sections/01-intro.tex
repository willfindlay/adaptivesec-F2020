\section{Introduction}

Containers offer a light-weight alternative to traditional hypervisor-based virtualization techniques by leveraging a thin virtualization layer provided by the operating system and sharing the operating system kernel and resources \cite{sultan2019_container_security,sun2018_security_namespace}. Thanks to their low overhead, high scalability, and composability, containers see increased industry adoption, particularly in multi-tenant platforms such as the cloud \cite{sultan2019_container_security}. Despite these performance and convenience advantages, containers suffer from weaker security than traditional virtualization techniques, in part because they must share the underlying operating system kernel \cite{sultan2019_container_security,sun2018_security_namespace,xin2018_container_security}. Another major factor impacting container security is the relatively lax attitude that many container management frameworks take toward least-privilege enforcement \cite{sultan2019_container_security}. Popular container frameworks such as Docker \cite{docker} provision overly-permissive default access, rely on a complex and often ill-suited suite of security mechanisms provided by the host system, and support insecure configuration options \cite{docker,sultan2019_container_security,xin2018_container_security,findlay2020_bpfbox}.

This paper proposes \bpfcontain{}, a novel approach to container security under the Linux kernel, to rectify overprivileged and insecure containers. Leveraging a new Linux kernel technology called eBPF, \bpfcontain{} uses runtime security instrumentation to implement container-specific policy enforcement and harden the host kernel against privilege escalation attacks mounted from containers. Specifically, \bpfcontain{} attaches eBPF programs to Linux's LSM (Linux Security Modules) hooks and critical functions within the kernel to enforce per-container policy in kernelspace. \bpfcontain{} combines user-defined policy files, implicit rules for container safety, and sensible defaults to enforce least-privilege access to sensitive system resources and secure the host-container boundary.

In summary, I offer the following contributions:
\begin{itemize}
  \item I present \bpfcontain{}, a novel least-privilege enforcement mechanism for containers, leveraging eBPF programs attached to LSM hooks in the kernel. \bpfcontain{} improves container security by offering a container-specific LSM that can be dynamically loaded at runtime. Using eBPF, it also integrates with parts of the kernel that are not directly covered by LSM hooks, hardening the host against kernel privilege escalation attacks mounted from a container. Thanks to eBPF's dynamic instrumentation capabilities, this integration occurs at runtime and requires no modification or patching of the kernel.

  \item I propose a simple YAML-based policy language that offers optional layers of granularity to meet various use cases and multiple stakeholders' needs. The policy language is expressive enough that it can be used to confine individual system resources, yet simple enough that it affords ad-hoc confinement use cases.

  \item Using the \bpfcontain{} prototype as a proof of concept, I argue that it can be integrated directly with existing container management frameworks without modifying their source code, offering significant security advantages over traditional approaches to container least-privilege. I present opportunities for future work that will allow \bpfcontain{} to both significantly streamline its policy language while simultaneously improving upon the status quo in confinement.
\end{itemize}

The rest of this paper proceeds as follows. \Cref{sec:background} presents background on process confinement, containers, container security, and eBPF. \Cref{sec:motivation} presents the container security threat model and discusses the motivation for implementing a new solution for container security. \Cref{sec:design} examines the design and implementation of \bpfcontain{}, including its policy language and enforcement mechanisms. \Cref{sec:eval} presents the methodology that will be used to conduct a formal evaluation of \bpfcontain{}'s usability, security, and performance. \Cref{sec:discussion} discusses limitations and opportunities for future work. \Cref{sec:related} covers related work and \Cref{sec:conclusion} concludes.
